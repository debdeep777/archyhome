\RequirePackage[l2tabu, orthodox]{nag}
\documentclass[12pt]{amsart}
%\usepackage{epsf}
%\usepackage{showlabels}
\usepackage{amssymb,latexsym, amsmath, amscd, array, graphicx}
\usepackage{amssymb}
\usepackage{amsmath}
\usepackage[a4paper]{geometry}

\usepackage{xcolor}
\definecolor{sage}{RGB}{0,204,153}
\usepackage{graphicx}
\usepackage{microtype}
\usepackage[colorlinks=false, pdfborder={0 0 0}]{hyperref} 
\usepackage{booktabs}
\usepackage{fancybox}
\usepackage{tikz}
\usetikzlibrary{shadows}
\usepackage[framemethod=TikZ]{mdframed}
\usepackage{lipsum}
\usepackage{titlesec}


\mdfdefinestyle{MyFrame}{%
    linecolor=sage,
    roundcorner=8pt,
    outerlinewidth=2pt,
    innertopmargin=\baselineskip,
    innerbottommargin=\baselineskip,
    innerrightmargin=20pt,
    innerleftmargin=20pt,
    shadow=true,}
    
\theoremstyle{plain}
\newtheorem{mainthm}{Theorem}
\newtheorem{thm}{Theorem}[section]
\newtheorem{theorem}[thm]{Theorem}
\newtheorem{lem}[thm]{Lemma}
\newtheorem{lemma}[thm]{Lemma}
\newtheorem{conjec}[thm]{Conjecture}
\newtheorem{prop}[thm]{Proposition}
\newtheorem{cor}[thm]{Corollary}
\newcommand\theoref{Theorem~\ref}
\newcommand\lemref{Lemma~\ref}
\newcommand\propref{Proposition~\ref}
\newcommand\corref{Corollary~\ref}
\newcommand\defref{Definition~\ref}
\newcommand\remref{Remark~\ref}
\newcommand\remsref{Remarks~\ref}


\def\secref{Section~\ref}
%\theoremstyle{remark}

\theoremstyle{definition}
\newtheorem{defin}[thm]{Definition}
\newtheorem{definition}[thm]{Definition}
\newtheorem{glos}[thm]{Glossary}
\newtheorem{defins}[thm]{Definitions}
\newtheorem{rem}[thm]{Remark}
\newtheorem{remark}[thm]{Remark}
\newtheorem{rems}[thm]{Remarks}
\newtheorem{ex}[thm]{Example}
\newtheorem{con}[thm]{Construction}
\newtheorem{conven}[thm]{Convention}
\newtheorem{notat}[thm]{Notation}
\newtheorem{setup}[thm]{Set up}
\newtheorem{question}[thm]{Question}
\newtheorem{problem}[thm]{Problem}

\titlespacing\section{0pt}{12pt plus 4pt minus 2pt}{0pt plus 2pt minus 2pt}
\titlespacing\subsection{0pt}{12pt plus 4pt minus 2pt}{0pt plus 2pt minus 2pt}
\titlespacing\subsubsection{0pt}{12pt plus 4pt minus 2pt}{0pt plus 2pt minus 2pt}



\makeatletter
\renewcommand{\paragraph}{%
\@startsection{paragraph}{4}%
{\z@}{0.80ex \@plus 1ex \@minus .2ex}{-1em}%
{\normalfont\normalsize\bfseries}%
}
\makeatother

\addtolength{\hoffset}{-1cm}
\addtolength{\textwidth}{2cm}

\begin{document}
\title{\textsc{Math 141 Exam 1 Review}}
\maketitle

\section*{Finding the Volume of a Surface of Revolution}
\begin{mdframed}[style=MyFrame]
\begin{center}
\textsc{\textbf{Quick Cheats}}
\begin{enumerate}
\item \textbf{Disk Method:} Usually used for functions revolved around $x$-axis.Every cross-section looks like a disk, hence the name.
\begin{align*}
V = \int\limits_{a}^{b}\pi [f(x)]^2 dx
\end{align*}
\item \textbf{Washer Method:} Used to compute volume of solid generated by revolving region between two functions $f$ and $g$. Guess what every cross section looks like? (A washer, obviously.) 
\begin{align*}
V = \int\limits_{a}^{b}\pi[\left( f(x) \right)^2-\left( g(x) \right)^2]dx
\end{align*}
\item \textbf{Cross Section Method:} Used to evaluate volume of a solid with cross section $A(x)$. 
\begin{align*}
V = \int\limits_{a}^{b}A(x) dx
\end{align*}
\item \textbf{Shell Method:} Usually used for regions  revolved around $y$-axis.Adding up the surface area of each of the shells to make the volume.
\begin{align*}
V = \int\limits_{a}^{b}2\pi x [f(x) -g(x)] dx
\end{align*}

\end{enumerate}
\end{center}
\end{mdframed}

\subsection*{\textsc{Four Methods}} 
\begin{enumerate}            
\item \textsc{Disk Method}
    \begin{itemize}
    \setlength{\itemsep}{200pt}
    \item[Ex1:] Find the volume of the solid created by rotating the region between the x-axis and the graph of $y=\sqrt{9 - x^2}$ on the interval $[1,2]$ about the x-axis. 
    \item[Ex2:] Let $R$ be the region between the $x$-axis and the graph of $\frac{x^2}{3^2} + \frac{y^2}{4^2} = 1$. What is the volume $V$ of the solid of revolution formed by rotating $R$ around the $x$-axis? 
    \item[] 
    \end{itemize}  
\item \textsc{Washer Method} 
    \begin{itemize} 
    \setlength{\itemsep}{200pt} 
    \item[Ex1:] Let R be the region bounded by the curves $y = x^4 $ and $y = x$, with $x\geq 0$. What is the volume $V$ of the solid of revolution formed by rotating $R$ around the $x$-axis?   
    \item[Ex2:]Let R be the region bounded by the curves $y = x^2 $ and $y = x$, with $x\geq 0$. What is the volume $V$ of the solid of revolution formed by rotating $R$ around the the line $x = -2$? 
    \item[] 
    \end{itemize} 

\item \textsc{Base and Cross Section}
    \begin{itemize}
    \setlength{\itemsep}{200pt}
    \item[Ex1:] Find the volume of the solid whose base is the elliptical region with boundary curve $9x^2 + 4x^2 = 36$ with each cross-section perpendicular to the $x$-axis a square.  
    \item[Ex2:] Let the base of a solid be a $30^\circ - 60^\circ$ right triangle, with smallest leg of length 5 unit. The cross section of the solid perpendicular to that leg are squares. Draw a picture of the base, and a cross section, and a cross section, and then find the volume $V$ of the solid. 
    \item[]
    \end{itemize}
    
\item \textsc{Shell Method}
    \begin{itemize}
    \setlength{\itemsep}{200pt}
    \item[Ex1:] Let $R$ be the region between the graph of $f(x) = (x-1)^2$ and the $x$-axis on the interval $[0,2]$. Use the Shell Method to find the volume $V$ of the solid generated by revolving $R$ about the $y$-axis.
    \item[Ex2:] Let $R$ be the region between the graph of $f(x) = 2^x$     and the $x$-axis on the interval $[1,4]$. Use the Shell Method to find the volume   $V$ of the solid generated by revolving $R$ about the $y$-axis.
    \item[]
    \end{itemize}

\end{enumerate}

\section*{Center of Gravity}              


\begin{mdframed}[style=MyFrame]
\begin{center}                 
\textbf{Quick Cheats}        
\end{center}                  
\end{mdframed}

\begin{enumerate}
\setlength{\itemsep}{200pt} 
\item[Ex1:] Find the centroid of the region bounded by the given curves, $y= e^x $, $y=0$, $x=0$, and $x=1$.               
\item[Ex2:] What are the coordinates, $(\overline{x},\overline{y})$, of the center of gravity of the region bounded by the line $3x +2y =6$ and the coordinate axes.               
\item[Ex3:] 
\item[]
\end{enumerate}



\section*{Work Done, Springs, Water and Whatnots...}             

\begin{mdframed}[style=MyFrame]
\begin{center}               
\textbf{Quick Cheats}          
\end{center}                   

\end{mdframed}

\begin{enumerate}
\setlength{\itemsep}{200pt}
\item[Ex1:] If it requires 9 ft-lb of work to stretch a spring 2 ft beyond its resting position, how much work is required to stretch the spring 4ft past its resting position?                  
\item[Ex2:] 
   \begin{enumerate}
   \item A bowl (in the shape of a hemisphere) with a radius of 10ft is filled with liquid nitrogen. What is the work required to empty one third of the liquid nitrogen to a height 1 ft above the bowl? For this question assume that liquid nitrogen weighs 50.2 pounds per cubic foot. 
   \item What if the liquid nitrogen was in a cone with the same radius as in part (a) with height of 8 ft (all other constraints are as in part (a))?
   \end{enumerate}
\item[Ex3:] 
\item[]
\end{enumerate}

\section*{Arc Length, Length of a Curve, or what have you.}              

\begin{mdframed}[style=MyFrame]
\begin{center}                 
\textbf{Quick Cheats}
\begin{enumerate}
\item 
\item The ``intuitive" idea behind the formula : 
\end{enumerate}
\end{center}                  

\end{mdframed}

\begin{enumerate}
\setlength{\itemsep}{200pt} 
\item[Ex1:] Let $f(x)= \frac{x^5}{7} + \frac{1}{9x^3}$. What is the length of the graph of $f$ on the interval $[1,2]$?     
\item[Ex2:] Suppose that a curve in the plane has the parametrization of $x= e^t + e^{-t}$ and $y= 5 -2t$ on the interval $[0,3]$. What is the exact length of the curve?                  
\item[Ex3:]   
\item[]  
\end{enumerate}  

\section*{Limits and L'Hospital}               

\begin{mdframed}[style=MyFrame]
\begin{center}                 
\textbf{Quick Cheats}          
\end{center}                   

\end{mdframed}
Find the following limits. 
\begin{enumerate}
\setlength{\itemsep}{200pt} 
\item[Ex1:] $\lim_{x to 2} \frac{\ln 3-x}{\sin \frac{1}{2}\pi x}$                
\item[Ex2:] $\lim_{x to \infty} \frac{e^x}{3x^3}$                      
\item[Ex3:] 
\item[]
\end{enumerate}

\section*{Differential equations}             

\begin{mdframed}[style=MyFrame]
\begin{center}               
\textbf{Quick Cheats}          
\end{center}                   

\end{mdframed}

\begin{enumerate}
\setlength{\itemsep}{200pt}
\item[Ex1:] Find the general solution of the differential equation $$$$                
\item[Ex2:]                 
\item[Ex3:] 
\item[]
\end{enumerate}

\section*{Topic 1}             

\begin{mdframed}[style=MyFrame]
\begin{center}               
\textbf{Quick Cheats}          
\end{center}                   

\end{mdframed}

\begin{enumerate}
\setlength{\itemsep}{200pt}
\item[Ex1:]                 
\item[Ex2:]                 
\item[Ex3:] 
\item[]
\end{enumerate}

\section*{Topic 1}             

\begin{mdframed}[style=MyFrame]
\begin{center}               
\textbf{Quick Cheats}          
\end{center}                   

\end{mdframed}

\begin{enumerate}
\setlength{\itemsep}{200pt}
\item[Ex1:]                 
\item[Ex2:]                 
\item[Ex3:] 
\item[]
\end{enumerate}
\end{document} 
